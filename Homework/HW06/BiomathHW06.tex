\documentclass[]{article}
\usepackage{lmodern}
\usepackage{amssymb,amsmath}
\usepackage{ifxetex,ifluatex}
\usepackage{fixltx2e} % provides \textsubscript
\ifnum 0\ifxetex 1\fi\ifluatex 1\fi=0 % if pdftex
  \usepackage[T1]{fontenc}
  \usepackage[utf8]{inputenc}
\else % if luatex or xelatex
  \ifxetex
    \usepackage{mathspec}
  \else
    \usepackage{fontspec}
  \fi
  \defaultfontfeatures{Ligatures=TeX,Scale=MatchLowercase}
\fi
% use upquote if available, for straight quotes in verbatim environments
\IfFileExists{upquote.sty}{\usepackage{upquote}}{}
% use microtype if available
\IfFileExists{microtype.sty}{%
\usepackage{microtype}
\UseMicrotypeSet[protrusion]{basicmath} % disable protrusion for tt fonts
}{}
\usepackage[margin=1in]{geometry}
\usepackage{hyperref}
\hypersetup{unicode=true,
            pdftitle={Biomath HW06},
            pdfauthor={Maxwell Greene},
            pdfborder={0 0 0},
            breaklinks=true}
\urlstyle{same}  % don't use monospace font for urls
\usepackage{graphicx,grffile}
\makeatletter
\def\maxwidth{\ifdim\Gin@nat@width>\linewidth\linewidth\else\Gin@nat@width\fi}
\def\maxheight{\ifdim\Gin@nat@height>\textheight\textheight\else\Gin@nat@height\fi}
\makeatother
% Scale images if necessary, so that they will not overflow the page
% margins by default, and it is still possible to overwrite the defaults
% using explicit options in \includegraphics[width, height, ...]{}
\setkeys{Gin}{width=\maxwidth,height=\maxheight,keepaspectratio}
\IfFileExists{parskip.sty}{%
\usepackage{parskip}
}{% else
\setlength{\parindent}{0pt}
\setlength{\parskip}{6pt plus 2pt minus 1pt}
}
\setlength{\emergencystretch}{3em}  % prevent overfull lines
\providecommand{\tightlist}{%
  \setlength{\itemsep}{0pt}\setlength{\parskip}{0pt}}
\setcounter{secnumdepth}{0}
% Redefines (sub)paragraphs to behave more like sections
\ifx\paragraph\undefined\else
\let\oldparagraph\paragraph
\renewcommand{\paragraph}[1]{\oldparagraph{#1}\mbox{}}
\fi
\ifx\subparagraph\undefined\else
\let\oldsubparagraph\subparagraph
\renewcommand{\subparagraph}[1]{\oldsubparagraph{#1}\mbox{}}
\fi

%%% Use protect on footnotes to avoid problems with footnotes in titles
\let\rmarkdownfootnote\footnote%
\def\footnote{\protect\rmarkdownfootnote}

%%% Change title format to be more compact
\usepackage{titling}

% Create subtitle command for use in maketitle
\newcommand{\subtitle}[1]{
  \posttitle{
    \begin{center}\large#1\end{center}
    }
}

\setlength{\droptitle}{-2em}

  \title{Biomath HW06}
    \pretitle{\vspace{\droptitle}\centering\huge}
  \posttitle{\par}
    \author{Maxwell Greene}
    \preauthor{\centering\large\emph}
  \postauthor{\par}
      \predate{\centering\large\emph}
  \postdate{\par}
    \date{March 18, 2020}


\begin{document}
\maketitle

\section{Problem \#1}\label{problem-1}

Non-Dimensionalize the following system of equations: \[
\begin{aligned}
\frac{dx}{dt} &= \alpha x - \beta xy \\
\frac{dy}{dt} &= \gamma xy - \delta y
\end{aligned}
\]

Start with setting non-dimensionalized values to arbitrary scalings of
the original, dimensioned values.

\[
\begin{aligned}
X & = \frac{x}{x_1} && \frac{dX}{dT} = \frac{dX}{dx}\frac{dx}{dt}\frac{dt}{dT} \\
Y & = \frac{y}{y_1} && \frac{dY}{dT} = \frac{dY}{dy}\frac{dy}{dt}\frac{dt}{dT} \\
\end{aligned}
\] This non-dimensionalized system becomes

\[
\begin{aligned}
\frac{dX}{dT} &= \Big( \frac{t_1}{x_1}\Big) \Big( \alpha x - \beta xy \Big) &&= \Big( \frac{t_1}{x_1}\Big) \Big( \alpha x_1X - \beta x_1Xy_1Y \Big) \\ 
& &&= \Big[ \alpha t_1 \Big] X - \Big[ \beta t_1 y_1 \Big] XY \\
\frac{dY}{dT} &= \Big( \frac{t_1}{y_1}\Big) \Big( \gamma xy - \omega y \Big) &&= \Big( \frac{t_1}{y_1}\Big) \Big( \gamma x_1X y_1Y - \beta y_1Y \Big) \\ 
& &&= \Big[ \gamma t_1 x_1 \Big] XY - \Big[ \omega t_1 \Big] Y
\end{aligned}
\] Now we must assign values to the parameters such that the bracketed
terms will be simplified.\\
The substitution \[
t_1 = \frac{1}{\alpha}, \quad
y_1 = \frac{\alpha}{\beta} \quad
x_1 = \frac{\alpha}{\gamma}
\] simplifies the parameters to this dimensionless system of equations:
\[
\begin{aligned}
\frac{dX}{dT} &= X-XY \\
\frac{dY}{dT} &= XY-aY, \quad \text{where } a=\frac{\gamma}{\alpha}
\end{aligned}
\]

\section{Problem \#2}\label{problem-2}

\subsection{(a)}\label{a}

\begin{verbatim}
## [1] "Tr(A)  > 0 :  FALSE   det(A) > 0 :  TRUE"
\end{verbatim}

\begin{verbatim}
## [1] "det(A) > (Tr(A)^2)/4 :  TRUE"
\end{verbatim}

\includegraphics{BiomathHW06_files/figure-latex/unnamed-chunk-3-1.pdf}

\subsection{(b)}\label{b}

\begin{verbatim}
## [1] "Tr(B)  > 0 :  FALSE   det(B) > 0 :  FALSE"
\end{verbatim}

\begin{verbatim}
## [1] "det(B) > (Tr(B)^2)/4 :  FALSE"
\end{verbatim}

\includegraphics{BiomathHW06_files/figure-latex/unnamed-chunk-5-1.pdf}

\subsection{(c)}\label{c}

\begin{verbatim}
## [1] "Tr(C)  > 0 :  FALSE   det(C) > 0 :  FALSE"
\end{verbatim}

\begin{verbatim}
## [1] "det(C) > (Tr(A)^2)/4 :  FALSE"
\end{verbatim}

\includegraphics{BiomathHW06_files/figure-latex/unnamed-chunk-7-1.pdf}

\subsection{(d)}\label{d}

\begin{verbatim}
## [1] "Tr(D)  > 0 :  FALSE   det(D) > 0 :  FALSE"
\end{verbatim}

\begin{verbatim}
## [1] "det(D) > (Tr(A)^2)/4 :  FALSE"
\end{verbatim}

\includegraphics{BiomathHW06_files/figure-latex/unnamed-chunk-9-1.pdf}

\section{Problem \#3}\label{problem-3}

\[
\begin{aligned}
\frac{dx}{dt} &= y-(x^2+y^2)x \\
\frac{dy}{dt} &= -x-(x^2+y^2)y
\end{aligned}
\]

\includegraphics{BiomathHW06_files/figure-latex/unnamed-chunk-10-1.pdf}


\end{document}
